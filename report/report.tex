\documentclass[a4paper,10pt]{article}

\usepackage[utf8]{inputenc}

% Title Page
\title{IMH Projet 2 : Compte Rendu}
\author{Saad-Dupuy Jean-Christophe, Carrier Geoffroy}


\begin{document}
\maketitle

%\begin{abstract}
%\end{abstract}


\section{Présentation du projet}
Ce projet porte sur le développement d'un programme de simulation du développement plantes dans un environnement défini par l'utilisateur.
\subsection{Contraintes}
\subsubsection{Interface}
L'utilisateur doit être capable d'ajouter une plante ou deux dans ce laboratore virtuel.
Il doit aussi pouvoir modifier différents paramètres afin de configurer l'environement pour
la semaine complète et voir les répercusions sur l'évolution de ces \textit{cobayes}.
dans lequel les plantes vont évoluer.

\subsubsection{Environnement}
L'environnement dans lequel évoluent les plantes est constitué des facteurs suivants :

\begin{enumerate}

\item \textbf{Plantes}

Nous disposons de  deux types de plantes :
\begin{enumerate}
 \item \textbf{Tomatito} 
\item \textbf{Cacai}
Ces plantes ont toutes deux trois états :
	\begin{itemize}
		\item la graine a germée (oui ou non),
		\item la plante s'est développée,
		\item le plante est morte.
	\end{itemize}
  
\end{enumerate}

\item \textbf{Fruit}
  .
Nos deux plantes peuvent aussi donner des fruits. Ces derniers sont spécifiques à chaque plantes et ont des propriéte partiulières, décrites ci-dessous :
\begin{enumerate}
 \item \textbf{Tomatito}
	\begin{enumerate}
		\item Couleur (vert, rouge, marron),
		\item Gout (amer, fade, acidulé)
	\end{enumerate}
 \item \textbf{Cacai}
		\begin{enumerate}
			\item Gout (sucre, amer),
			\item son diammetre
		\end{enumerate}
	\end{enumerate}

\item \textbf{La lumière}

L'utilisateur doit pouvoir régler la valeur de la lumière. D'après l'énoncé, les valeurs de celle-ci 
sont les suivantes :
	\begin{itemize}
	\item lumière directe = 1,
	\item lumière indirecte = 2,
	\item obscurité = 3.
	\end{itemize}

  \item \textbf{L'eau}

L'eau quand à elle est déterminée par deux éléments :
\begin{enumerate}
	\item Sa quantité, exprimée en centilitres,
	\item Sa dureté. Les valeurs possibles sonts les suivantes :
		\begin{itemize}
			\item Très douce (0),
			\item Douce (1),
			\item Moyennement dure (2),
			\item Dure (3),
			\item Très dure (4).
		\end{itemize}
	\end{enumerate}
De plus, l'utilisateur doit pouvoir configurer le nombre de jours d'arrosage (de 1 à 7).

 \item \textbf{terre}.
	\begin{enumerate}
	\item Sa quantité, exprimée en grammes,
	\item la présence d'engrais (oui ou non).
	\end{enumerate}
\end{enumerate}


\section{Conception}
Nous avons établi 
\subsection{Diagramme de classe}
\section{Rendu}
\end{document}          
