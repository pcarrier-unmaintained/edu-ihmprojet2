\documentclass[a4paper,10pt]{article}

\usepackage[utf8]{inputenc}

% Title Page
\title{IMH Projet 2 : Compte Rendu}
\author{Saad-Dupuy Jean-Christophe, Carrier Geoffroy}


\begin{document}
\maketitle

%\begin{abstract}
%\end{abstract}


\section{Présentation du projet}
Ce projet porte sur le développement d'un programme de simulation du développement plantes dans un environnement défini par l'utilisateur.
\subsection{Contraintes}
\subsubsection{Interface}
L'utilisateur doit être capable d'ajouter une plante ou deux dans ce laboratore virtuel.
Il doit aussi pouvoir modifier différents paramètres afin de configurer l'environement pour
la semaine complète et voir les répercusions sur l'évolution de ces \textit{cobayes}.
dans lequel les plantes vont évoluer.

\subsubsection{Environnement}
L'environnement dans lequel évoluent les plantes est constitué des facteurs suivants :

\begin{enumerate}

\item \textbf{Plantes}

Nous disposons de  deux types de plantes :
\begin{enumerate}
 \item \textbf{Tomatito} 
\item \textbf{Cacai}
Ces plantes ont toutes deux trois états :
	\begin{itemize}
		\item la graine a germée (oui ou non),
		\item la plante s'est développée,
		\item le plante est morte.
	\end{itemize}
  
\end{enumerate}

\item \textbf{Fruit}
  .
Nos deux plantes peuvent aussi donner des fruits. Ces derniers sont spécifiques à chaque plantes et ont des propriéte partiulières, décrites ci-dessous :
\begin{enumerate}
 \item \textbf{Tomatito}
	\begin{enumerate}
		\item Couleur (vert, rouge, marron),
		\item Gout (amer, fade, acidulé)
	\end{enumerate}
 \item \textbf{Cacai}
		\begin{enumerate}
			\item Gout (sucre, amer),
			\item son diammetre
		\end{enumerate}
	\end{enumerate}

\item \textbf{La lumière}

L'utilisateur doit pouvoir régler la valeur de la lumière. D'après l'énoncé, les valeurs de celle-ci 
sont les suivantes :
	\begin{itemize}
	\item lumière directe = 1,
	\item lumière indirecte = 2,
	\item obscurité = 3.
	\end{itemize}

  \item \textbf{L'eau}

L'eau quand à elle est déterminée par deux éléments :
\begin{enumerate}
	\item Sa quantité, exprimée en centilitres,
	\item Sa dureté. Les valeurs possibles sonts les suivantes :
		\begin{itemize}
			\item Très douce (0),
			\item Douce (1),
			\item Moyennement dure (2),
			\item Dure (3),
			\item Très dure (4).
		\end{itemize}
	\end{enumerate}
De plus, l'utilisateur doit pouvoir configurer le nombre de jours d'arrosage (de 1 à 7).

 \item \textbf{Terre}.
	\begin{enumerate}
	\item Sa quantité, exprimée en grammes,
	\item la présence d'engrais (oui ou non).
	\end{enumerate}
\end{enumerate}


\section{Conception}
\section{Modèle}
\subsection{Classes}
Nous avons choisi de créer une classe Simulation qui seras utilisés par la vue pour tous les appels au modèle.
\subsubsection{Classe Simulation}
Cette classe est la seule avec laquelle notre interface graphique va dialoguer.
C'est elle qui connait les deux types de plantes présentes, et fait la liaison avec avec l'environnement.

Elle à la particularité d'être un beans écoutable par l'interface graphique, mais aussi d'être un \textit{listener} 
sur les propriété de l'environnement.

Le diagramme UML de cette classe est donné ci-dessous :
% TODO inseérer diac SIMULATION.

Ses propriétées liées sont \textit{etatsPlante1} et \textit{etatsPlante2}, qui correspondent états des deux plantes
sur sept jours.


\subsubsection{Classe Environnement}
Toute la gestion des paramètres de l'environnement des plantes se fait dans la classse Environnement.
Cette classe contient en classe interne (\textit{inner-class}) les classes suivantes~:
\begin{itemize}
	\item Climat : gestion de la lumière et de la chaleur.
	\item Eau : Gestion de la quantité et de la dureté de l'eau,
	\item Terre : gestion de la quantité de terre et de l'engrais.
\end{itemize}

Cette classe, à partir des classes internes décrites ci-dessus calcule le taux de qualité de croissnce. Cette propriété peut être observée par l'interface graphique pour mettre à jour les plantes dans le rendu graphique.


\subsubsection{Classe Fruit}

Cette classe est la parente des fruits de type Tomatito et Cacai. Ces deux sous classes héritent donc de toutes les méthodes
définies dans Fruit et spécialisent cette classe en surchargeant les méthodes nécéssaires.

\subsection{Enumérations}

Pour faciliter notre développement, nous avons définis des enumérations pour les éléments suivants :
\begin{itemize}
 \item Couleur,
 \item Duretes,
 \item Goût,
 \item Type de plante.
\end{itemize}


\section{Vue}

Nous avons développée notre IHM de manière à ce qu'elle soit le plus claire possible.
Pour nous faciliter le travail, nous avons écrit notre propre widget permettant d'afficher les plante :

Notre interface graphique est la suivante :
\subsection{Diagramme de classe Général}
\section{Rendu}
\end{document}          
